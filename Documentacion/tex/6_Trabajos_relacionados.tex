\capitulo{6}{Trabajos relacionados}

El tutor proporcionó los enlaces de los repositorios de \textit{GitHub} de algunos proyectos similares a este en los que poder inspirarse. Dichos proyectos son también generadores de cuestionarios para \textit{Moodle}, pero de diferentes asignaturas y temas.

\section{PLQUIZ}
PLQUIZ\footnote{Roberto Izquierdo Amo, Julio 2014, Universidad de Burgos} es una herramienta de escritorio que permite generar preguntas de test aleatorias (tipo quiz, cloze, de texto libre...) sobre problemas de algoritmos de análisis léxico. El formato utilizado para generar las preguntas es .XML para importarse a entornos virtuales de aprendizaje (Moodle), y .TEX para  su impresión en papel a través de la obtención de código \LaTeX{}.\cite{plquiz}

Se han seguido sus pasos en el formato del traductor, al decidir crear una clase \textit{Plantilla} y tener las plantillas en formato .XML en el paquete \textit{resources}

\section{PLGRAM}
PLGRAM\footnote{Víctor Renuncio Tobar, Julio 2016, Universidad de Burgos} es una aplicación que a partir de una gramática obtiene el análisis sintáctico descendente \textit{LL}(1) y los análisis sintácticos ascendente \textit{SLR}(1), \textit{LR}(1 )y \textit{LALR}(1). Se obtienen los conjuntos \textit{FIRST} y \textit{FOLLOW} así como la tabla de análisis sintáctico predictivo (\textit{TASP}) y las tablas de \textit{ACCIÓN y de IR A}. Asimismo se puede obtener para los distintos análisis la traza de una cadena.

Con la aplicación se puede generar un archivo .XML para importar a la plataforma \textit{Moodle} con la finalidad de generar de forma automática cuestionarios de análisis sintáctico. También se puede generar un fichero en formato \LaTeX{} que se puede utilizar para realizar pruebas escritas. Se puede obtener un documento en el que el cuestionario esté completo, con las respuestas correctas y otro documento en el que el cuestionario está vacío, pensado para ser respondido por un tercero.\cite{plgram}

\section{ProgDinQuiz}
ProgDinQuiz\footnote{Asier Alonso Morante, Septiembre 2016, Universidad de Burgos} es una herramienta de escritorio que ayuda a generar preguntas de test aleatorias (tipo quiz, cloze...) sobre tipos de problema de programación dinámica.\cite{progdinquiz}