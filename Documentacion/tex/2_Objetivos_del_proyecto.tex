\capitulo{2}{Objetivos del proyecto}

\section{Objetivos técnicos}
Los principales objetivos técnicos perseguidos mediante el desarrollo de la aplicación han sido:

\begin{enumerate}
    \item Desarrollar una Interfaz Gráfica de Usuario para facilitar el uso de la aplicación.
    \item Establecer los parámetros necesarios para generar los enunciados de las preguntas.
    \item Generar los conjuntos de datos aleatorios para las preguntas en base a los parámetros establecidos.
    \begin{itemize}
    \item Para la pregunta de \textit{Ampliación Item Sets} generar un conjunto de n-item sets aleatorios.
    \item Para las preguntas de \textit{Generación Reglas Asociación} y \textit{Generación Item Sets} generar un conjunto de datos aleatorios que pueden tomar los valores T o F y en ocasiones, valores numéricos.
    \end{itemize}
    \item Generar los enunciados de las preguntas en base a los conjuntos de datos generados.
    \item Generar opciones verdaderas y falsas para las preguntas en base a los conjuntos de datos generados.
    \begin{itemize}
    \item Para las preguntas de \textit{Ampliación Item Sets} las opciones consistirán en n+1-item sets generados a partir de un conjunto de n-item sets.
    \item Para las preguntas de \textit{Generación Reglas Asociación} las opciones consistirán en reglas de asociación generadas a partir de un conjunto de datos.
    \item Para las preguntas de \textit{Generación Item Sets} las opciones consistirán en item sets generados a partir de un conjunto de datos.
    \end{itemize}
    \item Establecer una puntuación determinada para cada opción que se calcule automáticamente dependiendo de la cantidad de opciones que sean verdaderas y falsas.
    \item Generar preguntas a partir de los enunciados y opciones que se han generado.
    \item Generar un banco de preguntas de un tipo concreto a partir de las preguntas que se han obtenido.
    \item Traducir ese banco de preguntas a un formato .XML importable en Moodle.
    \item Exportar el fichero .XML en una ruta especificada y con un nombre especificado.
\end{enumerate}

\section{Objetivos académicos}
Algunos de los objetivos académicos que se han conseguido son los siguientes:

\begin{enumerate}
    \item Aplicar los conocimientos adquiridos en la carrera para desarrollar una aplicación práctica.
    \item Mejorar las habilidades de programación.
    \item Aprender a generar preguntas y respuestas automáticamente en el ámbito de minería de datos.
    \item Entender y aplicar los conceptos de item sets y reglas de asociación en la minería de datos.
    \item Comprender el proceso de discretización de atributos numéricos en los conjuntos de datos.
    \item Adquirir experiencia en la creación de ficheros en formato MoodleXML y su importación en el sistema de gestión de aprendizaje Moodle.
    \item Adquirir nuevos conocimientos y habilidades en herramientas y tecnologías relacionadas con la ingeniería informática.
    \item Mejorar las habilidades en presentación y comunicación, así como en la defensa y justificación del trabajo realizado.
\end{enumerate}

\section{Objetivos personales}
Los objetivos personales que se han llevado a cabo han sido los siguientes:

\begin{enumerate}
    \item Adquirir experiencia práctica en el desarrollo de un proyecto.
    \item Mejorar habilidades de resolución de problemas y pensamiento lógico.
    \item Desarrollar habilidades para la generación automática de preguntas y respuestas.
    \item Aumentar la confianza en las propias habilidades técnicas y académicas.
    \item Aprender a trabajar de manera independiente y a cumplir con un plazo establecido.
\end{enumerate}


