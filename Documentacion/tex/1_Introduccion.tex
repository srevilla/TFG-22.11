\capitulo{1}{Introducción}

La enseñanza en línea ha experimentado un gran aumento en la actualidad, especialmente en el ámbito universitario. Esto se ve reflejado en la creciente demanda de educación a distancia. Un ejemplo evidente es UBUVirtual, la plataforma virtual de la Universidad de Burgos, basada en Moodle, que es ampliamente utilizada. Por lo tanto, es esencial proporcionar las herramientas necesarias para mejorar la experiencia de estudiantes y profesores. Esto incluye la capacidad de crear y publicar exámenes en la plataforma. 

El proyecto presentado consiste en una aplicación de escritorio que permite generar cuestionarios en formato .XML sobre reglas de asociación para poseriormente ser importados en Moodle.

Las preguntas que se pueden generar mediante esta aplicación son las siguientes:

\begin{enumerate}
    \item \textbf{\textit{Generación Item Sets}}: dado un conjunto de datos con una cantidad determinada de atributos y transacciones obtener sus item sets con soporte mayor o igual que un valor determinado.

    En la siguiente imagen se muestra un ejemplo de pregunta de los apuntes de la asignatura de Minería de Datos que sirvió como inspiración para esta pregunta.
    
    \imagen{Pregunta3}{Inspiración pregunta \textit{Generación Item Sets}}{.7}
    
    \item \textbf{\textit{Generación Reglas Asociación}}: dado un conjunto de datos con una cantidad determinada de atributos e iteraciones, obtener las reglas de asociación con soporte y confianza mayores o iguales que unos valores específicos.
    
    En la siguiente imagen se muestra un ejemplo de pregunta de los apuntes de la asignatura de Minería de Datos que sirvió como inspiración para esta pregunta.
        
    \imagen{Pregunta2}{Inspiración pregunta \textit{Generación Reglas Asociación}}{.8}
    
    \item \textbf{\textit{Ampliación Item Sets}}: dado un conjunto que contiene un numero determinado de n-item sets, calcular los posibles n+1-item sets.
    
    En la siguiente imagen se muestra un ejemplo de pregunta de los apuntes de la asignatura de Minería de Datos que sirvió como inspiración para esta pregunta.
        
    \imagen{Pregunta1}{Inspiración pregunta \textit{Ampliación Item Sets}}{.8}
\end{enumerate}