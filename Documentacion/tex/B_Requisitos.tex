\apendice{Especificación de Requisitos}

\section{Introducción}

Este apéndice es una sección fundamental para el éxito de cualquier proyecto de software. En este apartado se describen los objetivos generales y se hace un catálogo exhaustivo de los requisitos necesarios para lograr esos objetivos. La especificación de requisitos es el proceso de identificar y documentar los requisitos de software que deben ser satisfechos para garantizar que el proyecto cumpla con las expectativas deseadas.

La importancia de esta sección radica en que es la base para el diseño, desarrollo, y pruebas del software. Si los requisitos no están claramente definidos, es probable que el proyecto se desvíe de su objetivo original y que los resultados finales no sean satisfactorios. Por lo tanto, la especificación de requisitos es un paso crítico en el desarrollo de software de alta calidad y es esencial para garantizar el éxito del proyecto. \cite{wiki:requisitos}

\section{Objetivos generales}

Los objetivos generales del proyecto son los siguientes:

\begin{enumerate}
\item Generación de bancos de preguntas que contengan preguntas generadas de los siguientes tipos: \textit{Generación Item Sets}, \textit{Generación Reglas de Asociación} y \textit{Ampliación Item Sets}.
\item Configuración de diferentes parámetros para la generación de los bancos de preguntas, como el número de preguntas, el número de item sets o atributos, y otros, por el usuario.
\item Capacidad de establecer valores de forma aleatoria si el usuario no define algún parámetro en la configuración.
\item Obtención de un archivo .XML compatible con la plataforma \textit{Moodle} de forma que contenga todas las preguntas generadas de un mismo tipo.
\end{enumerate}

\section{Catálogo de requisitos}\label{catalogo-de-requisitos}

A continuación, se enumeran los requisitos específicos derivados de los objetivos generales del proyecto.

\subsection{Requisitos funcionales}\label{requisitos-funcionales}

\begin{itemize}
\tightlist
\item
  \textbf{RF-1 Generación de banco de preguntas:} la aplicación tiene que ser capaz de generar bancos de preguntas sobre reglas de asociación para la asignatura de minería de datos. Un banco de preguntas contiene preguntas que a su vez, tienen titulo, enunciado y opciones.

  \begin{itemize}
  \tightlist
  \item
    \textbf{RF-1.1 Generar banco de preguntas \textit{Ampliación Item Sets}:} La aplicación debe ser capaz de generar un banco de preguntas que contenga todas las preguntas generadas de este tipo.

    \begin{itemize}
    \tightlist
    \item
      \textbf{RF-1.1.1: Definir el número de preguntas a generar:} el usuario tiene que ser capaz de configurar el número de preguntas de este tipo que quiere generar.
    \item
      \textbf{RF-1.1.2: Definir el número de item sets:} el usuario tiene que poder definir el número de item sets que formarán parte del conjunto de n-item sets que va a ser generado.
    \item 
      \textbf{RF-1.1.3: Definir el tamaño de los item sets:} el usuario tiene que poder definir el tamaño de los item sets que formarán parte del conjunto de n-item sets que va a ser generado.
    \item
      \textbf{RF-1.1.4: Establecer la configuración de forma aleatoria:} el usuario no tiene que estar obligado a definir la configuración de todos los parámetros anteriores, de forma que si no establece un valor determinado para un parámetro, este valor se generará de forma aleatoria.
    \end{itemize}
  \item
    \textbf{RF-1.2 Generar banco de preguntas \textit{Generación Reglas de Asociación}:} La aplicación debe ser capaz de generar un banco de preguntas que contenga todas las preguntas generadas de este tipo.

    \begin{itemize}
    \tightlist
    \item
      \textbf{RF-1.2.1: Definir el número de preguntas a generar:} el usuario tiene que ser capaz de configurar el número de preguntas de este tipo que quiere generar.
    \item
      \textbf{RF-1.2.2: Definir el número de atributos:} el usuario tiene que poder definir el número de atributos que formarán parte del conjunto de datos que va a ser generado.
    \item
      \textbf{RF-1.2.3: Definir el número de instancias:} el usuario tiene que poder definir el número de instancias que formarán parte del conjunto de datos que va a ser generado.
    \item
      \textbf{RF-1.2.4: Definir el soporte mínimo:} el usuario tiene que poder definir el valor del soporte mínimo del problema.
    \item
      \textbf{RF-1.2.5: Definir la confianza mínima:} el usuario tiene que poder definir el valor de la confianza mínima del problema.
    \item
      \textbf{RF-1.2.6: Especificar si quiere atributos numéricos para ser discretizados:} el usuario tiene que poder establecer si quiere atributos numéricos para que sean discretizados.
    \item
      \textbf{RF-1.2.7: Definir el número de intervalos a discretizar:} el usuario tiene que poder definir el número de intervalos en los que se va a discretizar el atributo numérico.
    \item
      \textbf{RF-1.2.8: Establecer la configuración de forma aleatoria:} el usuario no tiene que estar obligado a definir la configuración de todos los parámetros anteriores, de forma que si no establece un valor determinado para un parámetro, este valor se generará de forma aleatoria.
    \end{itemize}
\item
    \textbf{RF-1.3 Generar banco de preguntas \textit{Generación Item Sets}:} La aplicación debe ser capaz de generar un banco de preguntas que contenga todas las preguntas generadas de este tipo.

    \begin{itemize}
    \tightlist
    \item
      \textbf{RF-1.3.1: Definir el número de preguntas a generar:} el usuario tiene que ser capaz de configurar el número de preguntas de este tipo que quiere generar.
    \item
      \textbf{RF-1.3.2: Definir el número de atributos:} el usuario tiene que poder definir el número de atributos que formarán parte del conjunto de datos que va a ser generado.
    \item
      \textbf{RF-1.3.3: Definir el número de instancias:} el usuario tiene que poder definir el número de instancias que formarán parte del conjunto de datos que va a ser generado.
    \item
      \textbf{RF-1.3.4: Definir el soporte mínimo:} el usuario tiene que poder definir el valor del soporte mínimo del problema.
    \item
      \textbf{RF-1.3.5: Definir la confianza mínima:} el usuario tiene que poder definir el valor de la confianza mínima del problema.
    \item
      \textbf{RF-1.3.6: Especificar si quiere atributos numéricos para ser discretizados:} el usuario tiene que poder establecer si quiere atributos numéricos para que sean discretizados.
    \item
      \textbf{RF-1.3.7: Definir el número de intervalos a discretizar:} el usuario tiene que poder definir el número de intervalos en los que se va a discretizar el atributo numérico.
    \item
      \textbf{RF-1.3.8: Establecer la configuración de forma aleatoria:} el usuario no tiene que estar obligado a definir la configuración de todos los parámetros anteriores, de forma que si no establece un valor determinado para un parámetro, este valor se generará de forma aleatoria.
    \end{itemize}
   \end{itemize}

\iffalse
\item
  \textbf{RF-2 Resolución de problemas:} la aplicación tiene que ser capaz de resolver los problemas que ella misma ha generado.
  \begin{itemize}
  \tightlist
  \item
    \textbf{RF-2.1 Resolución para el problema de calcular item sets:} la aplicación tiene que ser capaz de obtener todos los posibles n+1-item sets a partir de un conjunto de n-item sets y clasificarlos en válidos y no válidos.
  \item
    \textbf{RF-2.2 Resolución para el problema de obtener las reglas de asociación:} la aplicación tiene que ser capaz de obtener varias reglas de asociación a partir de un conjunto de datos y clasificarlas en válidas y no válidas.
  \end{itemize}
\fi 
\item
  \textbf{RF-2 Exportación de problemas:} la aplicación tiene que ser capaz de exportar todos los problemas que ha generado.

  \begin{itemize}
  \tightlist
    \item
    \textbf{RF-2.1 Introducir un nombre para el fichero:} el usuario tiene que poder introducir el nombre que desea dar a su fichero que va a generar.
    \item 
    \textbf{RF-2.2 Especificación de la ruta:} el usuario tiene que poder especificar la ruta en la que quiere que se guarde su fichero .XML generado.
  \item
    \textbf{RF-2.3 Exportación en formato Moodle XML:} la aplicación tiene que ser capaz de exportar todos los problemas generados en un fichero .XML importable en la plataforma Moodle.
  \end{itemize}

\item
    \textbf{RF-3 Obtener información de la aplicación:} el usuario tiene que poder obtener información acerca de la aplicación.

\end{itemize}

\subsection{Requisitos no funcionales}\label{requisitos-no-funcionales}
\begin{itemize}
\tightlist
\item
  \textbf{RNF-1 Usabilidad:} la aplicación debe ser fácil de usar, con una interfaz gráfica sencilla de forma que no requiera aprendizaje.
\item
  \textbf{RNF-2 Rendimiento:} la aplicación tiene que ser capaz de hacer sus funciones de forma rápida y sin estancarse.
\item
  \textbf{RNF-3 Escalabilidad:} la aplicación tiene que
  estar diseñada para añadir nuevas funcionalidades de forma sencilla.
\item
  \textbf{RNF-4 Disponibilidad:} la aplicación debe de estar disponible para su uso en todo momento, de forma que no dependa de una conexión a internet.
\item
  \textbf{RNF-5 Mantenibilidad}: la aplicación debe soportar cambios y mejoras sin que su funcionamiento general se vea afectado.
\item
  \textbf{RNF-6 Fiabilidad}: la aplicación debe de funcionar de manera confiable y tiene que ser capaz de recuperarse ante fallos no deseados.
  
\end{itemize}

\section{Especificación de requisitos}
En esta sección se muestra el diagrama de casos de uso resultante y se desarrollar cada uno de ellos.

\begin{landscape}
\imagenAncho{DiagramaCasosUso}{Diagrama de casos de uso}{1.5}
\end{landscape}

\subsection{Actores}
Solo interactuará con el sistema un actor, que en este caso será un profesor.

\subsection{Casos de uso}

\begin{table}[h]
	\centering
	\begin{tabularx}{\linewidth}{ p{0.21\columnwidth} p{0.71\columnwidth} }
		\toprule
		\textbf{CU-1}    & \textbf{Acerca de}\\
		\toprule
		\textbf{Versión}              & 1.0    \\
		\textbf{Autor}                & Sergio Revilla Merino \\
		\textbf{Requisitos asociados} & RF-3 \\
		\textbf{Descripción}          & Permite al usuario obtener información de la aplicación. \\
		\textbf{Precondición}         & Iniciar la aplicación. \\
		\textbf{Acciones}             &
		\begin{enumerate}
			\def\labelenumi{\arabic{enumi}.}
			\tightlist
			\item Pulsar el botón de \textit{Acerca de}.
		\end{enumerate} \\
		\textbf{Postcondición}        & Se abre un panel con información de la aplicación. \\
		\textbf{Excepciones}          & - \\
		\textbf{Importancia}          & Media \\
		\bottomrule
	\end{tabularx}
	\caption{CU-1 Acerca de.}
\end{table}

\begin{table}[h]
	\centering
	\begin{tabularx}{\linewidth}{ p{0.21\columnwidth} p{0.71\columnwidth} }
		\toprule
		\textbf{CU-2}    & \textbf{Ampliación Item Sets}\\
		\toprule
		\textbf{Versión}              & 1.0    \\
		\textbf{Autor}                & Sergio Revilla Merino \\
		\textbf{Requisitos asociados} & RF-1, RF-1.1, RF-1.1.1, RF-1.1.2, RF-1.1.3, RF-1.1.4 \\
		\textbf{Descripción}          & Permite al usuario acceder a una ventana de configuración de este tipo de pregunta. \\
		\textbf{Precondición}         & Iniciar la aplicación. \\
		\textbf{Acciones}             &
		\begin{enumerate}
			\def\labelenumi{\arabic{enumi}.}
			\tightlist
            \item Pulsar el botón
		\end{enumerate} \\
		\textbf{Postcondición}        & Se abre la ventana para configurar este tipo de pregunta. \\
		\textbf{Excepciones}          & - \\
		\textbf{Importancia}          & Alta \\
		\bottomrule
	\end{tabularx}
	\caption{CU-2 Ampliación Item Sets.}
\end{table}

\begin{table}[h]
	\centering
	\begin{tabularx}{\linewidth}{ p{0.21\columnwidth} p{0.71\columnwidth} }
		\toprule
		\textbf{CU-3}    & \textbf{Generación Reglas Asociación}\\
		\toprule
		\textbf{Versión}              & 1.0    \\
		\textbf{Autor}                & Sergio Revilla Merino \\
		\textbf{Requisitos asociados} & RF-1, RF-1.2, RF-1.2.1, RF-1.2.2, RF-1.2.3, RF-1.2.4, RF-1.2.5, RF-1.2.6, RF-1.2.7, RF-1.2.8 \\
		\textbf{Descripción}          & Permite al usuario acceder a una ventana de configuración de este tipo de pregunta.  \\
		\textbf{Precondición}         & Iniciar la aplicación \\
		\textbf{Acciones}             &
		\begin{enumerate}
			\def\labelenumi{\arabic{enumi}.}
			\tightlist
            \item Pulsar el botón
		\end{enumerate} \\
		\textbf{Postcondición}        & Se abre la ventana para configurar este tipo de pregunta. \\
		\textbf{Excepciones}          & - \\
		\textbf{Importancia}          & Alta \\
		\bottomrule
	\end{tabularx}
	\caption{CU-3 Generación Reglas Asociación.}
\end{table}

\begin{table}[h]
	\centering
	\begin{tabularx}{\linewidth}{ p{0.21\columnwidth} p{0.71\columnwidth} }
		\toprule
		\textbf{CU-4}    & \textbf{Generación Item Sets}\\
		\toprule
		\textbf{Versión}              & 1.0    \\
		\textbf{Autor}                & Sergio Revilla Merino \\
		\textbf{Requisitos asociados} & RF-1, RF-1.3, RF-1.3.1, RF-1.3.2, RF-1.3.3, RF-1.3.4, RF-1.3.5, RF-1.3.6, RF-1.3.7, RF-1.3.8 \\
		\textbf{Descripción}          & Permite al usuario acceder a una ventana de configuración de este tipo de pregunta.  \\
		\textbf{Precondición}         & Iniciar la aplicación. \\
		\textbf{Acciones}             &
		\begin{enumerate}
			\def\labelenumi{\arabic{enumi}.}
			\tightlist
            \item Pulsar el botón
		\end{enumerate} \\
		\textbf{Postcondición}        & Se abre la ventana para configurar este tipo de pregunta. \\
		\textbf{Excepciones}          & - \\
		\textbf{Importancia}          & Alta \\
		\bottomrule
	\end{tabularx}
	\caption{CU-4 Generación Item Sets.}
\end{table}

\begin{table}[h]
	\centering
	\begin{tabularx}{\linewidth}{ p{0.21\columnwidth} p{0.71\columnwidth} }
		\toprule
		\textbf{CU-5}    & \textbf{Configuración de Ampliación Item Sets}\\
		\toprule
		\textbf{Versión}              & 1.0    \\
		\textbf{Autor}                & Sergio Revilla Merino \\
		\textbf{Requisitos asociados} & RF-1, RF-1.1, RF-1.1.1, RF-1.1.2, RF-1.1.3, RF-1.1.4 \\
		\textbf{Descripción}          & Permite al usuario configurar una pregunta de este tipo.  \\
		\textbf{Precondición}         & Pulsar el botón de \textit{Ampliación Item Sets} \\
		\textbf{Acciones}             &
		\begin{enumerate}
			\def\labelenumi{\arabic{enumi}.}
			\tightlist
			\item Seleccionar un número de preguntas a generar.
            \item Seleccionar o no el tamaño del conjunto de item sets y el tamaño de los item sets.
		\end{enumerate} \\
		\textbf{Postcondición}        & - \\
		\textbf{Excepciones}          & - \\
		\textbf{Importancia}          & Alta \\
		\bottomrule
	\end{tabularx}
	\caption{CU-5 Configuración de Ampliación Item Sets.}
\end{table}

\begin{table}[h]
	\centering
	\begin{tabularx}{\linewidth}{ p{0.21\columnwidth} p{0.71\columnwidth} }
		\toprule
		\textbf{CU-6}    & \textbf{Configuración del conjunto de datos}\\
		\toprule
		\textbf{Versión}              & 1.0    \\
		\textbf{Autor}                & Sergio Revilla Merino \\
		\textbf{Requisitos asociados} & RF-1, RF-1.2, RF-1.2.1, RF-1.2.2, RF-1.2.3, RF-1.2.4, RF-1.2.5, RF-1.2.6, RF-1.2.7, RF-1.2.8, RF-1.3, RF-1.3.1, RF-1.3.2, RF-1.3.3, RF-1.3.4, RF-1.3.5, RF-1.3.6, RF-1.3.7, RF-1.3.8 \\
		\textbf{Descripción}          & Permite al usuario configurar las preguntas de \textit{Generación Reglas Asociación} y \textit{Generación Item Sets}.  \\
		\textbf{Precondición}         & Pulsar el botón de \textit{Generación Reglas Asociación} o el de \textit{Generación Item Sets} \\
		\textbf{Acciones}             &
		\begin{enumerate}
			\def\labelenumi{\arabic{enumi}.}
			\tightlist
			\item Seleccionar un número de preguntas a generar.
            \item Seleccionar o no el número de instancias a generar, el número de atributos del conjunto de datos, el valor del soporte mínimo, el valor de la confianza, si se quieren atributos numéricos y el número de intervalos.
		\end{enumerate} \\
		\textbf{Postcondición}        & - \\
		\textbf{Excepciones}          & - \\
		\textbf{Importancia}          & Alta \\
		\bottomrule
	\end{tabularx}
	\caption{CU-6 Configuración del conjunto de datos.}
\end{table}

\begin{table}[h]
	\centering
	\begin{tabularx}{\linewidth}{ p{0.21\columnwidth} p{0.71\columnwidth} }
		\toprule
		\textbf{CU-7}    & \textbf{Exportar XML}\\
		\toprule
		\textbf{Versión}              & 1.0    \\
		\textbf{Autor}                & Sergio Revilla Merino \\
		\textbf{Requisitos asociados} & RF-2, RF-2.1, RF-2.2, RF-2.3 \\
		\textbf{Descripción}          & Traduce y exporta el banco de preguntas en la ruta seleccionada.  \\
		\textbf{Precondición}         & Se ha pulsado el botón de \textit{Exportar} \\
		\textbf{Acciones}             &
		\begin{enumerate}
			\def\labelenumi{\arabic{enumi}.}
			\tightlist
            \item Darle un nombre al fichero que se va a generar.
            \item Seleccionar la ruta deseada.
            \item Pulsar el botón \textit{Guardar}
            \item Se genera el banco de preguntas en base a la configuración establecida.
            \item Se traduce el banco de preguntas a formato XML.
		\end{enumerate} \\
		\textbf{Postcondición}        & Se genera un fichero .XML en la ubicación establecida y con el nombre establecido.\\
		\textbf{Excepciones}          & No se ha introducido ningún nombre para el fichero (mensaje). \\
		\textbf{Importancia}          & Alta \\
		\bottomrule
	\end{tabularx}
	\caption{CU-7 Exportar XML.}
\end{table}

