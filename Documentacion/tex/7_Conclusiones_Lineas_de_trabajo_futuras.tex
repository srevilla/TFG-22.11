\capitulo{7}{Conclusiones y Líneas de trabajo futuras}

\section{Conclusiones}


En el desarrollo de este proyecto, no solo se ha logrado el objetivo de crear una herramienta útil para los profesores de la Universidad de Burgos, sino que también se han perfeccionado habilidades relacionadas con la programación y el desarrollo de software, al ser la primera vez que se hace un programa totalmente desde cero. Durante el desarrollo de la aplicación, se han puesto en práctica conocimientos adquiridos en materias relacionadas con la programación y se han adquirido nuevos conocimientos sobre la creación de interfases gráficas y la importación de archivos en plataformas externas, así como ampliar los conocimientos relacionados con la gestión de proyectos.

Además, la colaboración con el tutor ha permitido mejora habilidades relacionadas con la comunicación y la resolución de problemas.

En conclusión, el desarrollo de esta aplicación ha sido un proceso enriquecedor tanto en términos de conocimientos técnicos como personales. Se ha logrado el objetivo de crear una herramienta útil y eficiente para los profesores de la universidad, y se han perfeccionado habilidades relacionadas con la programación y la resolución de problemas.

\section{Líneas de trabajo futuras}

En este apartado se presentan algunas sugerencias y posibilidades para el desarrollo futuro del proyecto.

\begin{itemize}
    \item \textbf{Desarrollo de tests unitarios}: Una de las áreas en las que se podría trabajar es en la implementación de tests unitarios para garantizar la calidad del software y detectar posibles errores de manera temprana.
    \item \textbf{Adición de preguntas nuevas}: Se podría ampliar los tipos de preguntas que la aplicación puede generar para mejorar la experiencia del usuario.
    \item \textbf{Traducción a más lenguajes}: Se podrían exportar los bancos de preguntas generados en diferentes formatos además del .XML, ya que \textit{Moodle} acepta diferentes tipos de formatos.
    \item \textbf{Mejora de la interfaz gráfica de usuario}: Se podrían realizar mejoras en la interfaz gráfica del programa, de manera que sea más visual e intuitiva para el usuario. 
\end{itemize}

Estas son solo algunas ideas, pero existen muchas otras áreas en las que se podría trabajar para mejorar y ampliar el proyecto.